%% Преамбула TeX-файла

% 1. Стиль и язык
\documentclass[utf8x, 12pt]{G7-32} % Стиль (по умолчанию будет 14pt)

\usepackage{totcount}

\newtotcounter{citnum} %From the package documentation
\def\oldbibitem{} \let\oldbibitem=\bibitem
\def\bibitem{\stepcounter{citnum}\oldbibitem}

\usepackage{etoolbox}

\newcounter{totfigures}

\providecommand\totfig{}

\makeatletter
\AtEndDocument{%
  \addtocounter{totfigures}{\value{figure}}%
  \immediate\write\@mainaux{%
    \string\gdef\string\totfig{\number\value{totfigures}}%
  }%
}
\makeatother

\pretocmd{\chapter}{\addtocounter{totfigures}{\value{figure}}\setcounter{figure}{0}}{}{}


% Остальные стандартные настройки убраны в preamble.inc.tex.
\include{10-preamble}

\begin{document}

\frontmatter % выключает нумерацию ВСЕГО; здесь начинаются ненумерованные главы: реферат, введение, глоссарий, сокращения и прочее.
\begin{center}
РЕФЕРАТ
\end{center}

% \large Московский авиационный институт\\[5.5cm]

% \huge Реферат \\[0.6cm] % название работы, затем отступ 0,6см
% \large на тему:  <<Метод идентификации музыкальных
% произведений по аудио фрагментам концертных исполнений>>\\[3.7cm]


% \end{center}

% \begin{flushright}
% Выполнил: студент гр. М8О-406Б \\
% Давид Гринберг \\
% \end{flushright}


% \vfill

% \begin{center}
% \large Москва 2020
% \end{center}

% \thispagestyle{empty}
Выпускная квалификационная работа содержит \pageref{LastPage} страницу, \totfig{}
рисунков, \total{citnum}\ использованных источников.

АКУСТИЧЕСКИЙ ОТПЕЧАТОК, ИНФОРМАЦИОННЫЙ ПОИСК, МЕТОД ГЛАВНЫХ КОМПОНЕНТ, SHAZAM, МУЗЫКАЛЬНЫЙ АНАЛИЗ, ВРЕМЕННЫЕ РЯДЫ.

Выпускная квалификационная работа посвящена разработке библиотеки создания акустических оптечатков с помощью
алгоритма, основанного на методе главных компонент и решению с ее помощью задачи идентификации музыкальных произведений
по аудио фрагментам концертных исполнений.

В теоретической части рассматриваются общие подходы к хранению и поиску аудиофайлов, описывается алгоритм
создания акустических отпечатков. Также предлагаются способы оптимизации поиска аудифайлов.

В практической части рассматривается архитектура реализованной библиотеки, приводятся результаты замеров
ее эффективности. Кроме того, описывается дальнейшее развитие библиотеки.


\thispagestyle{empty}
\setcounter{page}{0}
\setcounter{tocdepth}{2}
\setcounter{secnumdepth}{2}
\tableofcontents
\clearpage


\Introduction

В современном мире наблюдаются следующие тенденции:
\begin{itemize}
    \item Люди нетерпеливы и привыкли к легкому и быстрому доступу к информации
    \item Количество доступной информации неукротимо растет и человек не в состоянии
    справиться с ее потоком без использования поисковиков
    \item Существенная часть информации -- аудиофайлы
\end{itemize}

Конкретный пример: люди, посещающие различные музыкальные мероприятия, часто сталкиваются
с ситуацией, когда на сцене выступает музыкант, а название песни
или даже имя исполнителя неизвестно (например, на фестивале).
Конечно, можно спросить ближайшего человека, но в таких местах обычно очень шумно.
Кроме того нет гарантий, что у этого человека найдется ответ на вопрос.
Существует множество методов и сервисов для нахождения музыкальных произведений по отрывку,
однако у них есть ряд ограничений:
\begin{itemize}
    \item Сервисы вроде Shazam способны распознавать только оригиналы
    \item Некоторые сервисы умеют искать произведения по мелодии, но у них
            довольно низкая точность.
    \item Сервисы, которые ищут каверы или ремиксы (для защиты авторских прав) не приспособлены
            к нахождению зашумленных отрывков, поскольку предполагают, что кавер записывался в
            студийных условиях
\end{itemize}
В этой работе рассмотрен метод, лишенный всех вышеобозначенных недостатков.
Поскольку музыкальных произведений в мире очень много (по некоторым оценкам около 97 миллионов песен),
то очень важно уметь быстро и эффективно по памяти обрабатывать эти данные.
Кроме того этот метод применим не только к песням, так как аудиофайл представляет собой временной ряд.\\
Цели данной работы:
\begin{enumerate}[label=\arabic*.]
    \item Разработать библиотеку, которая предоставляла бы гибкий и удобный интерфейс
            для эффективной обработки и поиска аудиофайлов
    \item Реализовать клиент для идентификации концертных записей, используя разработанную
    библиотеку
\end{enumerate}


\mainmatter

\chapter{Теоретическая часть}
\label{cha:ch_1}

\section{Физика звука}
Звук - это вибрация, которая распространяется через воздух (или воду).
Например, при прослушивании музыки с компьютера колонки производят вибрации,
которые распространяются по воздуху, пока не достигнут уха человека.

Вибрации можно смоделировать с помощью синусоидальных волн.

\subsection{Чистый тон}
Чистый тон - это тон синусоидальной формы волны. Характеристики синусоиды:
\begin{itemize}
    \item Частота: количество циклов в секунду. Единица измерения - Герц (Гц), например, 100 Гц = 100 циклов в секунду.
    \item Амплитуда (связана с громкостью звука): размер каждого цикла.
\end{itemize}

Эти характеристики расшифровываются человеческим ухом для формирования звука.
Человек может слышать чистые тоны от $20$ Гц до $20 000$ Гц,
и этот диапазон уменьшается с возрастом. Для сравнения, свет, который видит человек,
состоит из синусоид от $4 * 10^{14}$ Гц до $7.9 * 10^{14}$ Гц.

Человеческое восприятие громкости зависит от частоты чистого тона.
Например, чистый тон с амплитудой равной $10$ и частотой $30$ Гц будет тише,
чем чистый тон с амплитудой $10$ и частотой $1000$ Гц.
Человеческие уши воспринимают звук в соответствии с психоакустической моделью.

Чистых тонов в природе не существует, однако каждый звук в мире - это сумма
нескольких чистых тонов с разными амплитудами.

\subsection{Музыкальные ноты}
Ноты разделены на октавы. В большинстве западных стран октава представляет
собой набор из 8 нот (A, B, C, D, E, F, G в большинстве англоязычных
стран) со следующим свойством:
\begin{itemize}
    \item Частота ноты в октаве удваивается в следующей октаве.
    Например, частота А4 (А в 4-й октаве) на частоте 440 Гц в 2 раза
    превышает частоту А3 (А в 3-й октаве) на 220 Гц и в 4 раза больше
    частоты А2 (А во 2-й октаве) на 110 Гц.
\end{itemize}

% Для 4-й октавы ноты имеют следующую частоту:
% \begin{itemize}
%     \item C4 = 261.63 Гц
%     \item D4  = 293.67 Гц
%     \item E4 = 329.63 Гц
%     \item F4 = 349.23 Гц
%     \item G4 = 392 Гц
%     \item A4 = 440 Гц
%     \item B4 = 493.88 Гц
% \end{itemize}

Частотная чувствительность ушей логарифмическая. Это означает, что:
\begin{itemize}
    \item между 32.70 Гц и 61.74 Гц (1-я октава)
    \item или между 261.63 Гц и 466.16 Гц (4-я октава)
    \item или между 2 093 Гц и 3 951.07 Гц (7-я октава)
\end{itemize}

Человеческие уши распознают одинаковое количество нот.


\section{Техника акустического отпечатка}
Для того, чтобы эффективно хранить и искать аудиофайлы, нужно
найти какое-нибудь компактное представление, которое при этом будет
максимально правдоподобно их описывать.
Это представление называется акустическим отпечатком (фингерпринтом) аудиофайла.
Существует множество видов таких отпечатков, но большинство методов
находят представление аудиофайлов в виде вектора хешей.

Факторы эффективности:
\begin{enumerate}[label=\arabic*.]
    \item Хеши максимизируют произведение функций энтропии и точности:
    \begin{center}
        \includegraphics[scale=0.6]{inc/img/best_hp.png}
    \end{center}
    \item Биты хешей сбалансированы, декоррелированы и имеют высокую дисперсию
\end{enumerate}

\subsection{Общая идея}
Многие алгоритмы фингерпринтинга выглядят так:
\begin{enumerate}[label=\arabic*.]
    % \item Индексация:
    % \begin{enumerate}
        \item Посчитать спектрограмму аудиофайла
        \item Применить на ней какую-либо оконную функцию (спектрально-временные фильтры)
        \item Конвертировать результат в вектор хешей
    %     \item Посчитать обратный индекс вида:
    %     $$hash \to [... \{song\_id,\ offset\} ... ]$$, где $offset$ -- это
    %     номер временного диапазона аудиофайла, соответствующего $song\_id$,
    %     в котором встречается $hash$.
    % \end{enumerate}
    % \item Поиск:
    % \begin{enumerate}
    %     \item Посчитать акустический отпечаток запроса
    %     \item Поскольку
    % \end{enumerate}
\end{enumerate}

\section{Метод хешпринтов}
Этот метод предложен в \cite{tsai}. Он, как и многие другие, находит представление
аудиофайла в виде вектора хешей.

Метод отличается следующими характеристиками:
\begin{enumerate}[label=\arabic*.]
    \item Обучение без учителя
    \item Высокая адаптивность к данным
    \item Независимость от силы сигнала (громкости звука)
\end{enumerate}

Самой важной отличительной чертой метода является обучение без учителя.
Такие методы, как, например, Chromaprint, описанный в \cite{chromaprint}, используют
заранее подготовленные спектрально-временные фильтры.
Метод хешпринтов находит эти фильтры непосредственно при индексации, что позволяет
ему учитывать специфику данных.
\begin{figure}
    \begin{center}
        \includegraphics[scale=0.3]{inc/img/chroma.png}
        \caption{Фильтры, используемые Chromaprint}
    \end{center}
\end{figure}

\subsection{Алгоритм вычисления хешпринта}
Для вычисления хешпринта, содержащего $N$ бит, нужно проделать следующее:
\begin{enumerate}[label=\arabic*.]
    \item Посчитать спектрограмму.\\
    Результат этапа: матрица $Spectrogram \in \mathbb{R}^{B \times n}$, где $B$ -- количество частотных диапазонов,
    $n$ -- количество временных диапазонов.
    \item Собрать контекстные фреймы полученной спектрограммы.
    Фреймы рассчитываются следующим образом:
    $$frame_i = V_{i-w}...V_{i+w}$$, где $V_i$ -- столбец спектрограммы, $w$ -- количество столбцов контекста.\\
    Результат этапа: матрица $Frames \in \mathbb{R}^{Bw \times n}$
    \item Применить к фреймам спектрально-временные фильтры. Фильтры представляют собой
    $N \times Bw$ матрицу и расситываются c помощью алгоритма обучения без учителя
    путем решения задачи оптимизации.\\
    Результат этапа: матрица признаков $Features \in \mathbb{R}^{N \times n}$.
    \item Посчитать дельту -- изменение признаков в течение промежутка $T$.
    Дельта рассчитывается по формуле:
    $$\Delta_i = feature_i - feature_{i+T}$$
    \item Наложить функцию порога и упаковать признаки в хешпринты:
    $$hashprint_i = intN(\Delta_i > 0)$$
\end{enumerate}

\subsection{Вычисление спектрально-временных фильтров}
Фильтры подбираются таким образом, чтобы признаки, полученные при их наложении,
имели максимальную дисперсию и в то же время были декоррелированы.\\
Для этого можно применить метод главных компонент (PCA):
\begin{enumerate}[label=\arabic*.]
    \item Посчитать ковариационные матрицы для всех матриц фреймов
    и просуммировать их.\\
    Результат этапа: матрица $CovarianceMatrix \in \mathbb{R}^{Bw \times Bw}$
    \item Найти $N$ собственных векторов с максимальными собственными значениями.\\
    Результат этапа: матрица $Filters \in \mathbb{R}^{N \times Bw}$
\end{enumerate}

\subsection{Специфика задачи идентификации живых отрывков}
Поскольку речь идет о нечетком поиске, то мы хотим учесть как можно
больше нюансов (признаков) сигнала. Поэтому будем представлять аудиофайлы в виде
64-битных хешпринтов. Также в качестве спектрограммы возьмем CQT спектрограмму -
она хороша тем, что ее частотные диапазоны можно подобрать таким образом, что они будут
соответствовать конкретным нотам.
Из-за того что мы имеем дело с пространством довольно большой размерности, мы не можем
использовать обратный индекс, поэтому поиск будет выглядеть примерно так:
\begin{enumerate}[label=\arabic*.]
    \item Для каждого оригинала из базы: прикладываем к нему отрывок и ищем такой отступ, чтобы сумма расстояний
    Хемминга между соответствующими хешпринтами была минимальной.
    \begin{center}
        \includegraphics[scale=0.5]{inc/img/query.png}
    \end{center}
    \item Собираем результаты, сортируем и возвращаем top-N
\end{enumerate}

\chapter{Практическая часть}
\label{cha:ch_2}

В рамках ВКР была реализована
библиотека hpfw (\href{https://github.com/kisasexypantera94/hpfw}{\color{blue} ссылка} на Github).
С использованием инструментов библиотеки была решена задача идентификации музыкальных
произведений по фрагментам концертных исполнений. Для этой же
задачи написан Telegram-бот (\href{https://t.me/hpfw_bot}{\color{blue} ссылка} на бота).

\section{Технологии}
В библиотеке hpfw используются:
\begin{itemize}
    \item C++17
    \item essentia -- для вычисления спектрограмм
    \item Eigen3 -- для линейной алгебры
    \item cpp-taskflow -- для распараллеливания индексации
\end{itemize}

Для библиотеки также написан Python-клиент.

\section{Архитектура библиотеки}
В центре библиотеки два класса -- $Collector$ (коллектор) и $HashprintHandle$ (хеншпринт-хендл).
Эти классы связаны паттерном <<Стратегия>>. Хендлы предоставляют инструменты (функции) для вычисления
хешпринтов. Коллекторы используют эти инструменты по своему усмотрению и занимаются непосредственно
вычислением хешпринтов. Благодаря такой структуре, можно будет легко тестировать различные способы
распараллеливания индексации.

\section{Клиент}
Пока что не очень ясно, где стоит проводить черту между библиотекой и клиентом. На данный момент
клиент использует только класс $Collector$.

\section{Telegram-бот}
Пример работы:
\begin{figure}[H]
    \centering
    \includegraphics[scale=0.6]{inc/img/tgbot.png}
    \caption{}
\end{figure}

\section{Результаты}
Замеры проводились на Intel i5-6360U (4) @ 2.00GHz, 8GB RAM
\begin{itemize}
    \item На индексацию одного трека уходит в среднем 5.7 секунд
    \item На полный поиск отрывка по базе из 167 треков (без индекса) уходит 4 миллисекунды
    \item Точность поиска 216 9-секундных отрывков по базе из 167 треков -- 0.78.
    \item Одна спектрограмма занимает около 10 МБ (правда, их не всегда нужно хранить)
\end{itemize}

Стоит отметить, что поиск осуществлялся без знания исполнителя, то есть это оценка в худшем случае.

\backmatter %% Здесь заканчивается нумерованная часть документа и начинаются ссылки и
            %% заключение

\Conclusion % заключение к отчёту

Результаты ВКР:
\begin{enumerate}[label=\arabic*.]
    \item Реализована библиотека для создания акустических отпечатков
    \item С помощью инструментов библиотеки решена задача идентификации музыкальных
    произведений по аудио фрагментам концертных исполнений
    \item Исследованы некоторые способы оптимизации хранения и поиска акустических отпечатков
\end{enumerate}

Рассмотренный метод хешпринтов имеет множество преимуществ по сравнению
с другими алгоритмами создания акустических отпечатков.
Кроме того этот метод универсален относительно природы данных, что позволит
применить его в самых разных направлениях.

\nocite{*}
\bibliographystyle{gost780u}
\begin{thebibliography}{9}
    \bibitem{tsai} Tsai, T. (2016). Audio Hashprints: Theory \& Application.
    (Doctoral dissertation, EECS Department, University of California, Berkeley).

    \bibitem{chromaprint}
    Yan Ke, Derek Hoiem, Rahul Sukthankar. (2005). Computer Vision for Music
    Identification, Proceedings of Computer Vision and Pattern Recognition.

    \bibitem{hnsw} Leonid Boytsov and Bilegsaikhan Naidan (2013).
    Engineering Efficient and Effective Non-metric Space Library.
    In Similarity Search and Applications - 6th International Conference, SISAP 2013,
    A Coru\~na, Spain, October 2-4, 2013, Proceedings (pp. 280–293). Springer.

    \bibitem{essentia} Bogdanov, D., Wack N., Gómez E., Gulati S., Herrera P., Mayor O., et al. (2013).
    ESSENTIA: an Audio Analysis Library for Music Information Retrieval. International Society for Music Information Retrieval Conference (ISMIR'13). 493-498.

    \bibitem{cqt} Schörkhuber, C., Klapuri, A., Holighaus, N., \& Dörfler, M. (n.d.).
    A Matlab Toolbox for Efficient Perfect Reconstruction Time-Frequency Transforms with Log-Frequency
    Resolution

    \bibitem{eigen} Gael Guennebaud, Benoit Jacob, \& others. (2010). Eigen v3.

    \bibitem{tf} T. Huang, C. Lin, G. Guo and M. Wong, "Cpp-Taskflow: Fast Task-Based Parallel
    Programming Using Modern C++," 2019 IEEE International Parallel and Distributed Processing
    Symposium (IPDPS), Rio de Janeiro, Brazil, 2019, pp. 974-983, doi: 10.1109/IPDPS.2019.00105.
\end{thebibliography}

\appendix   % Тут идут приложения

% \include{61-appendix}

\end{document}
